\documentclass{article}
\usepackage[utf8]{inputenc}
\usepackage{amsmath}
\usepackage{amsfonts}
\usepackage{graphicx}
% \usepackage{gensymb}
\title{Discussion Assignment Unit 8\\
Math 1201 - College Algebra.
}
\author{Jasper Albert Nri}
\date{October 2021}


\begin{document}

\maketitle

\section*{PART 1}
\title\textbf{QUESTION}\\
How can De Moivre's theorem be described? what is the scope of this theorem?
\title\textbf{Solution}\\
The De Moivre theorem is a theorem that is used for finding the powers and roots of complex numbers.\\
"De Moivre's theorem states that, for a positive integer n, ${z^n}$ is found by raising the modulus to the nth power and multiplying the argument by n"(Abramson, 2017. p 821).
\\if ${z = r\cos\theta + i\sin\theta}$ is a complex number then, 
$${z^n = r^n(\cos(n\theta) + i\sin(n\theta))}$$
$${z^n = r^n cis(n\theta)}$$
where n is a positive integer.\\
The De moivre's theorem can also be modified to find the nth roots of a complex number.\\
"To find the nth root of a complex number in polar form, use the formula given as "(Abramson, 2017. p 822).

$${z^{\frac{1}{n}} = r^{\frac{1}{n}}\left[\cos\left(\frac{\theta}{n}+\frac{2k\pi}{n}\right)  +  i\sin\left(\frac{\theta}{n}+\frac{2k\pi}{n}\right)\right]}$$
"where k = 0,1,2,3,\dots,n-1. we add ${\frac{2k\pi}{n} \text{to} \frac{\theta}{n}}$ in order to obtain the periodic roots"(Abramson, 2017. p 822).
\\The scope of the De Moivre's theorem is roots of complex equations and the powers of complex equations(Abramson, 2017).

\section*{PART 2}
Give two Examples of roots\\
\title\textbf{ROOTS: Example 1}\\
Find the four fourth roots of ${16cis(120^{\circ})}$
\\\title\textbf{Solution}\\
using the nth root theorem, 
$${z^{\frac{1}{n}} = r^{\frac{1}{n}}\left[\cos\left(\frac{\theta}{n}+\frac{2k\pi}{n}\right)  +  i\sin\left(\frac{\theta}{n}+\frac{2k\pi}{n}\right)\right]}$$

$${z^{\frac{1}{4}} = 16^{\frac{1}{4}}\left[\cos\left(\frac{120}{4}+\frac{2k\pi}{4}\right)  +  i\sin\left(\frac{120}{4}+\frac{2k\pi}{4}\right)\right]}$$
k = 0, 1, 2, 3.\\
when k = 0
$${z^{\frac{1}{4}} = 16^{\frac{1}{4}}\left[\cos\left(\frac{120}{4}+\frac{2(0)\pi}{4}\right)  +  i\sin\left(\frac{120}{4}+\frac{2(0)\pi}{4}\right)\right]}$$
$${z^{\frac{1}{4}} = 2[\cos(30)  +  i\sin(30)]}$$
$${z^{\frac{1}{4}} = 2\left[\frac{\sqrt{3}}{2}  +  i\frac{1}{2}\right]}$$
$${z^{\frac{1}{4}} = \sqrt{3} + i}$$
when k = 1
$${z^{\frac{1}{4}} = 16^{\frac{1}{4}}\left[\cos\left(\frac{120}{4}+\frac{2(1)\pi}{4}\right)  +  i\sin\left(\frac{120}{4}+\frac{2(1)\pi}{4}\right)\right]}$$
$${z^{\frac{1}{4}} = 2[\cos(30+90)  +  i\sin(30+90)]}$$
$${z^{\frac{1}{4}} = 2[\cos(120)  +  i\sin(120)]}$$
$${z^{\frac{1}{4}} = 2\left[-\frac{1}{2} + i\frac{\sqrt{3}}{2}\right]}$$
$${z^{\frac{1}{4}} = -1 + i\sqrt{3}}$$
when k = 2
$${z^{\frac{1}{4}} = 16^{\frac{1}{4}}\left[\cos\left(\frac{120}{4}+\frac{2(2)\pi}{4}\right)  +  i\sin\left(\frac{120}{4}+\frac{2(2)\pi}{4}\right)\right]}$$
$${z^{\frac{1}{4}} = 2[\cos(30+180)  +  i\sin(30+180)]}$$
$${z^{\frac{1}{4}} = 2[\cos(210)  +  i\sin(210)]}$$
$${z^{\frac{1}{4}} = 2\left[-\frac{\sqrt{3}}{2}  -  i\frac{1}{2}\right]}$$
$${z^{\frac{1}{4}} = -\sqrt{3} - i}$$
when k = 3
$${z^{\frac{1}{4}} = 16^{\frac{1}{4}}\left[\cos\left(\frac{120}{4}+\frac{2(3)\pi}{4}\right)  +  i\sin\left(\frac{120}{4}+\frac{2(3)\pi}{4}\right)\right]}$$
$${z^{\frac{1}{4}} = 2[\cos(30+270)  +  i\sin(30+270)]}$$
$${z^{\frac{1}{4}} = 2[\cos(300)  +  i\sin(300)]}$$
$${z^{\frac{1}{4}} = 2\left[ \frac{1}{2} - i\frac{\sqrt{3}}{2}\right]}$$
$${z^{\frac{1}{4}} = 1 - i\sqrt{3}}$$
Therefore the roots of the complex equation ${16cis(120^{\circ})}$ are ${\sqrt{3} + i}$, ${-1+i\sqrt{3}}$, ${-\sqrt{3}-i}$ and ${1-i\sqrt{3}}$.\\
\\\title\textbf{ROOTS: Example 2}\\
Evaluate the cube root of z when ${z = 32cis(\frac{2\pi}{3})}$
\\\title\textbf{Solution}\\
using the nth root theorem, 
$${z^{\frac{1}{n}} = r^{\frac{1}{n}}\left[\cos\left(\frac{\theta}{n}+\frac{2k\pi}{n}\right)  +  i\sin\left(\frac{\theta}{n}+\frac{2k\pi}{n}\right)\right]}$$

$${z^{\frac{1}{3}} = 32^{\frac{1}{3}}\left[\cos\left(\frac{\frac{2\pi}{3}}{3}+\frac{2k\pi}{3}\right)  +  i\sin\left(\frac{\frac{2\pi}{3}}{3}+\frac{2k\pi}{3}\right)\right]}$$
k = 0, 1, 2.\\
when k = 0
$${z^{\frac{1}{3}} = 32^{\frac{1}{3}}\left[\cos\left(\frac{2\pi}{9}+\frac{2(0)\pi}{3}\right)  +  i\sin\left(\frac{2\pi}{9}+\frac{2(0)\pi}{3}\right)\right]}$$
$${z^{\frac{1}{3}} = 32^{\frac{1}{3}}\left[\cos\left(\frac{2\pi}{9}\right)  +  i\sin\left(\frac{2\pi}{9}\right)\right]}$$
$${z^{\frac{1}{3}} = 32^{\frac{1}{3}} cis\left(\frac{2\pi}{9}\right)}$$
when k = 1
$${z^{\frac{1}{3}} = 32^{\frac{1}{3}}\left[\cos\left(\frac{2\pi}{9}+\frac{2(1)\pi}{3}\right)  +  i\sin\left(\frac{2\pi}{9}+\frac{2(1)\pi}{3}\right)\right]}$$
$${z^{\frac{1}{3}} = 32^{\frac{1}{3}}\left[\cos\left(\frac{2\pi}{9}+\frac{2\pi}{3}\right)  +  i\sin\left(\frac{2\pi}{9}+\frac{2\pi}{3}\right)\right]}$$
$${z^{\frac{1}{3}} = 32^{\frac{1}{3}}\left[\cos\left(\frac{8\pi}{9}\right)  +  i\sin\left(\frac{8\pi}{9}\right)\right]}$$
$${z^{\frac{1}{3}} = 32^{\frac{1}{3}} cis\left(\frac{8\pi}{9}\right)}$$
when k = 2
$${z^{\frac{1}{3}} = 32^{\frac{1}{3}}\left[\cos\left(\frac{2\pi}{9}+\frac{2(2)\pi}{3}\right)  +  i\sin\left(\frac{2\pi}{9}+\frac{2(2)\pi}{3}\right)\right]}$$
$${z^{\frac{1}{3}} = 32^{\frac{1}{3}}\left[\cos\left(\frac{2\pi}{9}+\frac{4\pi}{3}\right)  +  i\sin\left(\frac{2\pi}{9}+\frac{4\pi}{3}\right)\right]}$$
$${z^{\frac{1}{3}} = 32^{\frac{1}{3}}\left[\cos\left(\frac{14\pi}{9}\right)  +  i\sin\left(\frac{14\pi}{9}\right)\right]}$$
$${z^{\frac{1}{3}} = 32^{\frac{1}{3}} cis\left(\frac{14\pi}{9}\right)}$$
But $${\sqrt[3]{32} = \sqrt[3]{8 \times 4}}$$
But $${\sqrt[3]{32} = 2\sqrt[3]{4}}$$
Therefore the three roots of the complex equation  ${z = 32cis(\frac{2\pi}{3})}$ are ${2\sqrt[3]{4} \text{ cis}(\frac{2\pi}{9})}$, ${2\sqrt[3]{4} \text{ cis}(\frac{8\pi}{9})}$, and ${2\sqrt[3]{4} \text{ cis}(\frac{14\pi}{9})}$.

\section*{PART 3}
Give two Examples of powers\\
\title\textbf{POWERS: Example 1}\\
find $z^3 $ when ${z = 5cis(45^{\circ})}$\\
\title\textbf{Solution}\\
Using De Moivre's Theorem
$${z^n = r^n(\cos(n\theta) + i\sin(n\theta))}$$
$${z^n = r^n cis(n\theta)}$$
$${z^3 = 5^3(\cos(3\times45) + i\sin(3\times45))}$$
$${z^3 = 125(\cos(135) + i\sin(135))}$$
$${z^3 = 125\text{ cis}(135^{\circ})}$$
\\\title\textbf{POWERS: Example 2}\\
find $z^2 $ when ${z = 3cis(120^{\circ})}$\\
\title\textbf{Solution}\\
Using De Moivre's Theorem
$${z^n = r^n(\cos(n\theta) + i\sin(n\theta))}$$
$${z^n = r^n cis(n\theta)}$$
$${z^2 = 3^2(\cos(2\times120) + i\sin(2\times120))}$$
$${z^2 = 9(\cos(240) + i\sin(240))}$$
$${z^2 = 9\text{ cis}(240^{\circ})}$$
\section*{References}
Abramson, J. (2017). \textit{Algebra and trigonometry}. OpenStax, TX: Rice University. Retrieved
from https://openstax.org/details/books/algebra-and-trigonometry
\end{document}