\documentclass{article}
\usepackage[utf8]{inputenc}
\usepackage{amsmath}
\usepackage{amsfonts}
\usepackage{graphicx}
% \usepackage{gensymb}
\title{Learning Journal Unit 8\\
Math 1201 - College Algebra.
}
\author{Jasper Albert Nri}
\date{October 2021}


\begin{document}

\maketitle

\section*{QUESTION 1}
Evaluate the cube root of ${z = 27 \text{cis}(240^{\circ})}$\\
Then raise them to the cube(i.e the roots)\\
\title\textbf{Solution}\\
$${z = 27 \text{cis}(240^{\circ})}$$
Using the nth root theorem,
$${z^{\frac{1}{n}} = r^{\frac{1}{n}}\left[\cos\left(\frac{\theta}{n}+\frac{2k\pi}{n}\right)  +  i\sin\left(\frac{\theta}{n}+\frac{2k\pi}{n}\right)\right]}$$
Finding the cube roots of z
$${z^{\frac{1}{3}} = 27^{\frac{1}{3}}\left[\cos\left(\frac{240}{3}+\frac{2k\pi}{3}\right)  +  i\sin\left(\frac{240}{3}+\frac{2k\pi}{3}\right)\right]}$$
k = 0, 1, 2.
\\When k = 0
$${z^{\frac{1}{3}} = 27^{\frac{1}{3}}\left[\cos\left(\frac{240}{3}+\frac{2(0)\pi}{3}\right)  +  i\sin\left(\frac{240}{3}+\frac{2(0)\pi}{3}\right)\right]}$$
$${z^{\frac{1}{3}} = 3\left[\cos\left(80\right)  +  i\sin\left(80\right)\right]}$$

Therefore,
$${z^{\frac{1}{3}} = 3cis(80)}$$
When k = 1
$${z^{\frac{1}{3}} = 27^{\frac{1}{3}}\left[\cos\left(\frac{240}{3}+\frac{2(1)\pi}{3}\right)  +  i\sin\left(\frac{240}{3}+\frac{2(1)\pi}{3}\right)\right]}$$
$${z^{\frac{1}{3}} = 3\left[\cos\left(80 + 120\right)  +  i\sin\left(80 + 120\right)\right]}$$
$${z^{\frac{1}{3}} = 3\left[\cos\left(200\right)  +  i\sin\left(200\right)\right]}$$

Therefore,
$${z^{\frac{1}{3}} = 3cis(200)}$$
When k = 2
$${z^{\frac{1}{3}} = 27^{\frac{1}{3}}\left[\cos\left(\frac{240}{3}+\frac{2(2)\pi}{3}\right)  +  i\sin\left(\frac{240}{3}+\frac{2(2)\pi}{3}\right)\right]}$$
$${z^{\frac{1}{3}} = 3\left[\cos\left(80 + 240\right)  +  i\sin\left(80 + 240\right)\right]}$$
$${z^{\frac{1}{3}} = 3\left[\cos\left(320\right)  +  i\sin\left(320\right)\right]}$$

Therefore,
$${z^{\frac{1}{3}} = 3cis(320)}$$
The 3 roots of the equation are given as ${3cis(80)}$, ${3cis(200)}$, and ${3cis(320)}$\\
\\\textbf{From the question we are required to raise the roots to the cube}\\
Raising the roots of the equation to the cube using De Moivre's formula,
$${z^n = r^n(\cos(n\theta) + i\sin(n\theta))}$$
Cubing First root
$${3cis(80)}$$ 
$${Root_{1} = 3^3(\cos(80\times3)+i\sin(80\times3))}$$ 
$${Root_{1} = 27(\cos(240)+i\sin(240))}$$ 
$${Root_{1} = 27cis(240) \text{ or } Root_{1} = 27cis\left(\frac{4\pi}{3}\right)}$$ 
Cubing Second root
$${3cis(200)}$$
$${Root_{2} = 3^3(\cos(200\times3)+i\sin(200\times3))}$$ 
$${Root_{2} = 27(\cos(600)+i\sin(600))}$$ 
$${Root_{2} = 27cis(600) \text{ or } Root_{2} = 27cis\left(\frac{10\pi}{3}\right)}$$ 
Cubing Third root
$${3cis(320)}$$
$${Root_{3} = 3^3(\cos(320\times3)+i\sin(320\times3))}$$ 
$${Root_{3} = 27(\cos(960)+i\sin(960))}$$ 
$${Root_{3} = 27cis(960) \text{ or } Root_{3} = 27cis\left(\frac{16\pi}{3}\right)}$$ 




\section*{QUESTION 2}
Evaluate $${\left[\sqrt[3]{3}\left(\frac{\sqrt{3}}{2}+\frac{i}{2}\right)\right]^{10}}$$
\title\textbf{Solution}\\
To solve this equation i will be making use of the De Moivre's theorem formula.
$${z^n = r^n(\cos(n\theta) + i\sin(n\theta))}$$
But we must find the value of $\theta$ first.
Rearranging the given equation in De Moivre's formula we have 
$${z^{10} = (\sqrt[5]{3})^{10}\left(\frac{\sqrt{3}}{2}+\frac{i}{2}\right)}$$
If we compare it to the original equation, we can see that $\cos(\theta) = \frac{\sqrt{3}}{2}$ and ${\sin(\theta) = \frac{1}{2}}$
Solving for theta we get,
$$\cos(\theta) = \frac{\sqrt{3}}{2}$$
$$\theta = cos^{-1}\left(\frac{\sqrt{3}}{2}\right)$$
$$\theta = \frac{\pi}{6}$$
Also For sine we have, 
$$\sin(\theta) = \frac{{1}}{2}$$
$$\theta = sin^{-1}\left(\frac{{1}}{2}\right)$$
$$\theta = \frac{\pi}{6}$$
Rewriting De Moivre's formula with the value of $\theta$, we have,
$${z^{10} = (\sqrt[5]{3})^{10}\left(\cos\left(\frac{{\pi}}{6} \times  10\right)+i\sin\left(\frac{{\pi}}{6}\times 10\right)\right)}$$
Evaluating,
$${z^{10} = {3}^{\frac{10}{5}}\left(\cos\left(\frac{{5\pi}}{3}\right)+i\sin\left(\frac{{5\pi}}{3}\right)\right)}$$
$${z^{10} = {3}^{2}\left(\left(\frac{{1}}{2}\right)+\left(-i\frac{{\sqrt{3}}}{2}\right)\right)}$$
$${z^{10} = 9\left(\frac{{1}}{2}-i\frac{{\sqrt{3}}}{2}\right)}$$
Therefore, the final answer is 
$${z^{10} = 9\left(\frac{{1}}{2}-i\frac{{\sqrt{3}}}{2}\right)}$$






\section*{QUESTION 3}
Find ${\frac{z_{1}}{z_{2}}}$ in polar form:\\
$z_{1} = 21\text{cis}(135^{\circ})$  $z_{2} = 3\text{cis}(75^{\circ})$\\
\title\textbf{Solution}\\
Using the formula
$${\frac{z_{1}}{z_{2}} = \frac{r_{1}}{r_{2}}(\cos(\theta_{1}-\theta_{2})+i\sin(\theta_{1}-\theta_{2}))}$$
Substituting
$${\frac{z_{1}}{z_{2}} = \frac{21}{3}(\cos(135-75)+i\sin(135-75))}$$
$${\frac{z_{1}}{z_{2}} = 7(\cos(60)+i\sin(60))}$$
$${\frac{z_{1}}{z_{2}} = 7\left(\frac{1}{2}+i\frac{\sqrt{3}}{2}\right)}$$
Therefore,
$${\frac{z_{1}}{z_{2}} = \frac{7}{2}+i\frac{7\sqrt{3}}{2}}$$




\section*{References}
Abramson, J. (2017). \textit{Algebra and trigonometry}. OpenStax, TX: Rice University. Retrieved
from https://openstax.org/details/books/algebra-and-trigonometry
\end{document}